%% abntex2.latex, v<VERSION> dudektria
%% Copyright 2012-2015 by abnTeX2 group at http://abntex2.googlecode.com/ 
%%
%% This work may be distributed and/or modified under the
%% conditions of the LaTeX Project Public License, either version 1.3
%% of this license or (at your option) any later version.
%% The latest version of this license is in
%%   http://www.latex-project.org/lppl.txt
%% and version 1.3 or later is part of all distributions of LaTeX
%% version 2005/12/01 or later.
%%
%% This work has the LPPL maintenance status `maintained'.
%% 
%% The Current Maintainer of this work is Felipe Schneider.
%% Further information are available on http://abntex2.googlecode.com/

%%%%%%%%%%%%%%%%%%%%%%%%%%%%% USO %%%%%%%%%%%%%%%%%%%%%%%%%%%%%%%%%%%%%%%%%%%%%%
%
% Este é um template para o pandoc compatível com abntex2
%
% É necessário chamar pandoc com pelo menos -V documentclass=abntex2 e
% --template=/caminho/absoluto/para/abntex2.latex
%
% Desta maneira, classes baseadas na abntex2 ainda podem ser usadas. O caminho
% absoluto é devido ao fato de o pandoc ter problemas com caminhos relativos.
%
% Todos os exemplos foram compilados com as seguintes opções (exceto quando
% eles mesmos dizem o contrário):
%
%   pandoc -V documentclass=abntex2 \
%          --template=/caminho/absoluto/para/abntex2.latex \
%          -SRs --normalize --filter=pandoc-citeproc \
%          -V lang=english,french,spanish,german,brazil -V papersize=a4paper \
%          -V fontsize=12pt -V classoption=twoside -V classoption=openright \
%          -V linkcolor=blue caminho/do/arquivo.md -o caminho/do/arquivo.pdf
%
% Os com _art.pdf no final receberam também -V classoption=article
% Trocando a extensão do arquivo final de .pdf para .latex gera-se o código
% fonte, para processamento adicional posterior
%
%%%%%%%%%%%%%%%%%%%%%%%%%%%%% METADATA %%%%%%%%%%%%%%%%%%%%%%%%%%%%%%%%%%%%%%%%%
%
% As seguintes metadatas relacionadas ao abntex2 estão disponíveis no template:
% - title:			título do trabalho (caso não haja capa será chamado
%					\maketitle)
% - date:			data
% - author:			autor(es) do trabalho
% - place:			local
% - institution:	instituição
% - preamble:		ver documentação do abntex2 (\preambulo{})
% - abstract:		texto do resumo
% - tags:			lista de palavras-chave (aparece se abstract for definido)
% - tagstitle:		nome para substituir "Palavras-chave" no texto
% - capa:			se true chama \imprimircapa
% - folhaderosto:	se true chama \imprimirfolhaderosto
% - tipotrabalho:	ver documentação do abntex2
%
%%%%%%%%%%%%%%%%%%%%%%%%%%%%% CITAÇÕES %%%%%%%%%%%%%%%%%%%%%%%%%%%%%%%%%%%%%%%%%
%
% São feitas pelo pandoc-citeproc (chame pandoc --filter=pandoc-citeproc).
% Isto quer dizer que você precisa chamar
%
%   \postextual
%   # Referências
%
% ou algo semelhante no final do seu markdown. Para citar use as metadatas
% bibliography e csl como descrito na documentação do pandoc.
%
\documentclass[12pt,a4paper,openright,twoside,]{abntex2}	% Opções da classe do documento:
			% 	fontsize,lang,papersize,classoption
\usepackage[T1]{fontenc}
\usepackage{lmodern}	% Latin Modern
\usepackage{amssymb,amsmath}	% Símbolos matemáticos (Pandoc)
\usepackage{ifxetex,ifluatex}	% Testes de processadores (Pandoc)
\usepackage{fixltx2e} % provides \textsubscript
\usepackage{indentfirst}	% Indenta o primeiro parágrafo
\usepackage{color}	% Controle de cores
\usepackage{relsize}

\definecolor{blue}{RGB}{41,5,195}	% Altera o aspecto do azul
% use upquote if available, for straight quotes in verbatim environments
\IfFileExists{upquote.sty}{\usepackage{upquote}}{}
\ifnum 0\ifxetex 1\fi\ifluatex 1\fi=0 % if pdftex
  \usepackage[utf8]{inputenc}	% Adição de inputenc
\else % if luatex or xelatex
  \ifxetex
    \usepackage{mathspec}
    \usepackage{xltxtra,xunicode}
  \else
    \usepackage{fontspec}
  \fi
  \defaultfontfeatures{Mapping=tex-text,Scale=MatchLowercase}
  \newcommand{\euro}{€}
\fi
% use microtype if available
\IfFileExists{microtype.sty}{\usepackage{microtype}}{}
\hypersetup{breaklinks=true,
            bookmarks=true,
            pdfauthor={Daniel Q. Miranda},
            pdftitle={Proposta de Trabalho - Implementação do Protocol HTTP2 na linguagem Scala},
            colorlinks=true,
            citecolor=blue,
            urlcolor=blue,
            linkcolor=blue,
            pdfborder={0 0 0}}
\urlstyle{same}  % don't use monospace font for urls
\setlength{\parindent}{1.3cm}	% Tamanho dos parágrafos
\setlength{\parskip}{0.2cm}	% Espaçamento entre parágrafos
\setcounter{secnumdepth}{0}

\titulo{Proposta de Trabalho - Implementação do Protocol HTTP2 na linguagem
Scala}	% Adicionando o título
\autor{Daniel Q. Miranda}	% Adicionando autor(es)
\orientador{Prof.~Dr.~Daniel Macêdo Batista}  % Adicionando orientadores(es)
\data{}	% Adicionando a data

\begin{document}
\maketitle	% Imprimindo um título, caso não seja usada a capa



{
\hypersetup{linkcolor=black}
\setcounter{tocdepth}{3}
\noindent\tableofcontents* % Adicionando sumário
\cleardoublepage
}

\textual	% Início do texto convertido do arquivo original
\section{Contexto}\label{contexto}

O recém finalizado protocolo HTTP/2 foi criado para suceder o mais
importante e utilizado protocolo de comunicação de aplicações em rede na
Internet, o HTTP, modernizando-o com melhorias de eficiência e segurança
para o futuro.

Diversas implementações preliminares do HTTP/2 acompanharam seu
desenvolvimento, em múltiplas linguagens e ecossistemas de programação,
inclusive para a plataforma Java, fazendo uso da linguagem homônima.
Existe, porém, uma certa desconexão dos métodos e técnicas desta e de
outras linguagens que fazem uso da Java Virtual Machine, em especial as
de paradigma funcional, que prezam princípios como imutabilidade,
separação de lógica e E/S, etc.

\section{Proposta}\label{proposta}

Me proponho a implementar uma biblioteca e/ou servidor HTTP/2 na
linguagem Scala. Preferencialmente buscarei trabalhar para que ela seja
integrada ao projeto Akka), muito popular framework de concorrência e
computação distribuída na JVM, que recentemente iniciou um projeto de
implementação de HTTP combinando esforços de diversos outros projetos e
frameworks de aplicações Web. Caso isso não seja possível, a
desenvolverei de maneira independente.

\section{Aspectos de Estudo}\label{aspectos-de-estudo}

\begin{itemize}
\tightlist
\item
  Funcionamento do protocolo HTTP/2, incluindo novos desenvolvimentos
  como compressão
\item
  Estudo da implementação eficiente de protocolos e servidores de rede
\item
  Aplicação e avaliação de técnicas de
  *\href{http://www.reactivemanifesto.org/}{Reactive Programming}() na
  eficiência de comunicação, em especial na plataforma JVM
\end{itemize}

\section{Cronograma Aproximado}\label{cronograma-aproximado}

\begin{itemize}
\tightlist
\item
  Mar-Abr: Estudo preliminar, estudo do projeto Akka e conversa com
  desenvolvedores para verificar viabilidade de integração
\item
  Abr-Maio: Design inicial de arquitetura, planejamento, estudo do
  protocolo
\item
  Maio-Jul: Design de APIs, detalhamento da arquitetura, prototipação
  inicial
\item
  Jul-Out: Implementação de biblioteca já funcional, incluindo maior
  partes das features possível
\item
  Múltiplos streams (multiplexing)\\
\item
  Compressão de Headers\\
\item
  \emph{Server Push}\\
\item
  HTTP2c (TLS)\\
\item
  Out-Nov: Otimizações de performances, testes e comparações com outras
  implementações, elaboração de estudos de arquitetura, integração com
  algum software pré-existente para testes
\end{itemize}

\hyperdef{}{references}{\label{references}}	% No próprio texto é necessário invocar \postextual

\end{document}
