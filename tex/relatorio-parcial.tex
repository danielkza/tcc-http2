%% abntex2.latex, v<VERSION> dudektria
%% Copyright 2012-2015 by abnTeX2 group at http://abntex2.googlecode.com/ 
%%
%% This work may be distributed and/or modified under the
%% conditions of the LaTeX Project Public License, either version 1.3
%% of this license or (at your option) any later version.
%% The latest version of this license is in
%%   http://www.latex-project.org/lppl.txt
%% and version 1.3 or later is part of all distributions of LaTeX
%% version 2005/12/01 or later.
%%
%% This work has the LPPL maintenance status `maintained'.
%% 
%% The Current Maintainer of this work is Felipe Schneider.
%% Further information are available on http://abntex2.googlecode.com/

%%%%%%%%%%%%%%%%%%%%%%%%%%%%% USO %%%%%%%%%%%%%%%%%%%%%%%%%%%%%%%%%%%%%%%%%%%%%%
%
% Este é um template para o pandoc compatível com abntex2
%
% É necessário chamar pandoc com pelo menos -V documentclass=abntex2 e
% --template=/caminho/absoluto/para/abntex2.latex
%
% Desta maneira, classes baseadas na abntex2 ainda podem ser usadas. O caminho
% absoluto é devido ao fato de o pandoc ter problemas com caminhos relativos.
%
% Todos os exemplos foram compilados com as seguintes opções (exceto quando
% eles mesmos dizem o contrário):
%
%   pandoc -V documentclass=abntex2 \
%          --template=/caminho/absoluto/para/abntex2.latex \
%          -SRs --normalize --filter=pandoc-citeproc \
%          -V lang=english,french,spanish,german,brazil -V papersize=a4paper \
%          -V fontsize=12pt -V classoption=twoside -V classoption=openright \
%          -V linkcolor=blue caminho/do/arquivo.md -o caminho/do/arquivo.pdf
%
% Os com _art.pdf no final receberam também -V classoption=article
% Trocando a extensão do arquivo final de .pdf para .latex gera-se o código
% fonte, para processamento adicional posterior
%
%%%%%%%%%%%%%%%%%%%%%%%%%%%%% METADATA %%%%%%%%%%%%%%%%%%%%%%%%%%%%%%%%%%%%%%%%%
%
% As seguintes metadatas relacionadas ao abntex2 estão disponíveis no template:
% - title:			título do trabalho (caso não haja capa será chamado
%					\maketitle)
% - date:			data
% - author:			autor(es) do trabalho
% - place:			local
% - institution:	instituição
% - preamble:		ver documentação do abntex2 (\preambulo{})
% - abstract:		texto do resumo
% - tags:			lista de palavras-chave (aparece se abstract for definido)
% - tagstitle:		nome para substituir "Palavras-chave" no texto
% - capa:			se true chama \imprimircapa
% - folhaderosto:	se true chama \imprimirfolhaderosto
% - tipotrabalho:	ver documentação do abntex2
%
%%%%%%%%%%%%%%%%%%%%%%%%%%%%% CITAÇÕES %%%%%%%%%%%%%%%%%%%%%%%%%%%%%%%%%%%%%%%%%
%
% São feitas pelo pandoc-citeproc (chame pandoc --filter=pandoc-citeproc).
% Isto quer dizer que você precisa chamar
%
%   \postextual
%   # Referências
%
% ou algo semelhante no final do seu markdown. Para citar use as metadatas
% bibliography e csl como descrito na documentação do pandoc.
%
\documentclass[12pt,a4paper,openright,twoside,]{abntex2}	% Opções da classe do documento:
			% 	fontsize,lang,papersize,classoption
\usepackage[T1]{fontenc}
\usepackage{lmodern}	% Latin Modern
\usepackage{amssymb,amsmath}	% Símbolos matemáticos (Pandoc)
\usepackage{ifxetex,ifluatex}	% Testes de processadores (Pandoc)
\usepackage{fixltx2e} % provides \textsubscript
\usepackage{indentfirst}	% Indenta o primeiro parágrafo
\usepackage{color}	% Controle de cores
\usepackage{relsize}

\definecolor{blue}{RGB}{41,5,195}	% Altera o aspecto do azul
% use upquote if available, for straight quotes in verbatim environments
\IfFileExists{upquote.sty}{\usepackage{upquote}}{}
\ifnum 0\ifxetex 1\fi\ifluatex 1\fi=0 % if pdftex
  \usepackage[utf8]{inputenc}	% Adição de inputenc
\else % if luatex or xelatex
  \ifxetex
    \usepackage{mathspec}
    \usepackage{xltxtra,xunicode}
  \else
    \usepackage{fontspec}
  \fi
  \defaultfontfeatures{Mapping=tex-text,Scale=MatchLowercase}
  \newcommand{\euro}{€}
\fi
% use microtype if available
\IfFileExists{microtype.sty}{\usepackage{microtype}}{}
\hypersetup{breaklinks=true,
            bookmarks=true,
            pdfauthor={Daniel Q. Miranda},
            pdftitle={Relatório Parcial - Implementação do Protocol HTTP2 na linguagem Scala},
            colorlinks=true,
            citecolor=blue,
            urlcolor=blue,
            linkcolor=blue,
            pdfborder={0 0 0}}
\urlstyle{same}  % don't use monospace font for urls
\setlength{\parindent}{1.3cm}	% Tamanho dos parágrafos
\setlength{\parskip}{0.2cm}	% Espaçamento entre parágrafos
\setcounter{secnumdepth}{0}

\titulo{Relatório Parcial - Implementação do Protocol HTTP2 na linguagem Scala}	% Adicionando o título
\autor{Daniel Q. Miranda}	% Adicionando autor(es)
\orientador{Prof.~Dr.~Daniel Macêdo Batista}  % Adicionando orientadores(es)
\data{}	% Adicionando a data

\begin{document}
\maketitle	% Imprimindo um título, caso não seja usada a capa



{
\hypersetup{linkcolor=black}
\setcounter{tocdepth}{3}
\noindent\tableofcontents* % Adicionando sumário
\cleardoublepage
}

\textual	% Início do texto convertido do arquivo original
\section{Proposta}\label{proposta}

Me proponho a implementar uma biblioteca e/ou servidor HTTP/2 na
linguagem Scala. Preferencialmente buscarei trabalhar para que ela seja
integrada ao projeto \href{http://akka.io}{Akka}, muito popular
framework de concorrência e computação distribuída na JVM, que
recentemente iniciou um projeto de implementação de HTTP combinando
esforços de diversos outros projetos e frameworks de aplicações Web.
Caso isso não seja possível, a desenvolverei de maneira independente.

\section{Cronograma previsto
inicialmente}\label{cronograma-previsto-inicialmente}

\begin{itemize}
\tightlist
\item
  Mar-Abr: Estudo preliminar, estudo do projeto Akka e conversa com
  desenvolvedores para verificar viabilidade de integração
\item
  Abr-Maio: Design inicial de arquitetura, planejamento, estudo do
  protocolo
\item
  Maio-Jul: Design de APIs, detalhamento da arquitetura, prototipação
  inicial
\item
  Jul-Out: Implementação de biblioteca já funcional, incluindo maior
  partes das features possível:

  \begin{itemize}
  \tightlist
  \item
    Múltiplos streams (multiplexing)\\
  \item
    Compressão de Headers\\
  \item
    \emph{Server Push}\\
  \item
    HTTP2c (TLS)\\
  \end{itemize}
\item
  Out-Nov: Otimizações de performances, testes e comparações com outras
  implementações, elaboração de estudos de arquitetura,
\end{itemize}

\section{Progresso atual}\label{progresso-atual}

Até o presente momento a maior parte do progresso constituiu em
compreender e avaliar:

\begin{itemize}
\tightlist
\item
  O protocolo HTTP/2\\
\item
  Implementações existentes do HTTP/2 em diversas linguagens\\
\item
  Implementações de outros servidores HTTP em Java\\
\item
  O ecossistema Scala e aplicação possível do resultado pela comunidade
\end{itemize}

\section{Plataforma Akka}\label{plataforma-akka}

O plano inicial era trabalhar dentro da plataforma Akka, já muito
adotada para concorrência em Java e Scala, e que parece estar, com
sucesso, iniciando um processo de unificação de bibliotecas de
comunicação na JVM.

O contato inicial do projeto foi recebido até com interesse, mas os
desenvolvedores manifestaram falta de recursos e/ou tempo para *** a
iniciativa. O Akka, assim como diversos outros grandes projetos em
Scala, vem sendo financiado pela empresa
\href{http://typesafe.com}{Typesafe}, e a maioria dos contribuidores de
grande-escala são seus empregados, com objetivos e planos pré-existentes
e nem sempre flexíveis.

Decidi então trabalhar por contra própria, tomando a biblioteca HTTP do
Akka apenas como inspiração. Porém, ainda cogito utilizar a biblioteca
homônima ao projeto, que lida principalmente com concorrência através do
modelo de atores (mais sobre isso segue nas próximas seções), entrada e
saída a nível de arquivos e sockets, dentre outros.

\section{O protocolo HTTP/2}\label{o-protocolo-http2}

A nova versão do protocolo HTTP
(\href{(https://tools.ietf.org/html/rfc7540}{HTTP/2}) adiciona um certo
nível de complexidade ao original para alcançar diversos objetivos,
visando modernizar o protocolo para que ele continue eficiente nas
décadas que seguirão.

Alguns exemplos destes objetivos, com seus correspondentes mecanismos,
são:

\subsubsection{Utilizar eficientemente um número pequeno (menor
possível) de
conexões}\label{utilizar-eficientemente-um-nuxfamero-pequeno-menor-possuxedvel-de-conexuxf5es}

O protocolo HTTP/2 permite a existência de múltiplos \emph{streams} de
dados, tanto cliente-servidor como na direção oposta, para aproveitar ao
máximo uma única conexão TCP. Isso permite que, por exemplo, diversos
recursos sejam carregadas simultaneamente, que ações de um usuário em um
aplicativo Web sejam enviadas sem espera, ou que um modo de uso com
troca de informações contínua seja possível (que hoje, por exemplo, é
comummente implementado em browsers por um segundo protocolo, o
\href{https://www.websocket.org/}{WebSocket}).

\subsubsection{Diminuir o uso de banda em
geral}\label{diminuir-o-uso-de-banda-em-geral}

Mecanismos de compressão são suportados universalmente pelo HTTP/1 há
mais de uma década, mas em múltiplos casos fraquezas de segurança (CRIME
(Kelsey 2002), BREACH (Prado, Harris, and Gluck 2013)) foram descobertas
durante seu uso, causadas pelo gradual ``vazamento'' de informações
sobre padrões dos dados ou meta-dados transmitidos. O HTTP/2 visa
melhorar significativamente esta situação introduzindo um mecanismo
próprio de compressão de headers, eficiente e resistente a esses
vazamentos, chamado
\href{https://http2.github.io/http2-spec/compression.html}{HPACK}.

\subsubsection{Melhorar eficiência de aplicações que usam o HTTP, como a
Web}\label{melhorar-eficiuxeancia-de-aplicauxe7uxf5es-que-usam-o-http-como-a-web}

O mecanismo de ``Server Push'', onde um servidor pode preventivamente
recomendar que um cliente faça uma requisição pode acelerar
significativamente a visualização de páginas Web em geral. Também tem
potencial benefício para o uso do HTTP como API (através de padrões como
REST e SOAP), permitindo o paginamento de recursos de maneira limpa e
eficiente.

\subsubsection{\texorpdfstring{Manter compatibilidade ``bidirecional''
entre HTTP/1 e
HTTP/2}{Manter compatibilidade bidirecional entre HTTP/1 e HTTP/2}}\label{manter-compatibilidade-bidirecional-entre-http1-e-http2}

Múltiplos mecanismos de negociação foram especificados, para permitir
que:

\begin{itemize}
\tightlist
\item
  Clientes pré-existentes se mantenham funcionando sem perda de
  funcionalidade
\item
  Clientes novos possam se conectar de maneira genérica a um servidor, e
  utilizar o HTTP/2 se possível
\item
  Seja possível identificar se um servidor é compatível com HTTP/2 sem
  overhead.
\end{itemize}

Os principais escolhidos e especificados no protocolo são:

\begin{itemize}
\tightlist
\item
  Negociação de upgrade: uma conexão HTTP/1 é iniciada, com um conjunto
  de headers informando a compatibilidade com HTTP/2. Isso permite
  conexão universal, aceita tanto por HTTP/1 e HTTP/2
\item
  Negociação de protocolo de aplicação através do TLS, quando em uso: o
  protocolo Transport Layer Security tem um mecanismo interno de
  negociação, que pode ser usado para selecionar HTTP/2 caso possível
\item
  Conexão direta através da versão 2, que tem um preâmbulo de conexão
  próprio, facilmente distinguível do HTTP/1
\end{itemize}

\section{Implementações existentes do HTTP/2 e outros servidores em
Java}\label{implementauxe7uxf5es-existentes-do-http2-e-outros-servidores-em-java}

As principais implementações no ecossistema Java vem de dois projetos já
muito estabelecidos: \href{http://eclipse.org/jetty}{Jetty} (do projeto
Eclipse) e \href{http://netty.io}{Netty}. Ambas tem alto foco em
performance e qualidade em geral, e tem grandes gamas de contribuidores
e suporte.

Após o início do projeto, descobri a existência do projeto
\href{https://github.com/http4s/blaze}{http4s-blaze} em Scala, com
objetivos similares aos meus, que implementa um servidor HTTP/2 já na
linguagem Scala. Teria sido interessante saber de sua existência de
antemão, mas não vejo um problema: o objetivo principal do trabalho é o
aprendizado e a experiência. Minha implementação pode seguir escolhas
diferentes de arquitetura, prioridades, etc.

\section{Decisões de design e escolhas
tomadas}\label{decisuxf5es-de-design-e-escolhas-tomadas}

Embora infelizmente ainda tenha produzido pouco código, não suficiente
para um protótipo, já tenho decididas diversas escolhas de modo de
trabalho, arquitetura, objetivos, etc.

\begin{itemize}
\tightlist
\item
  Sistema de build: \href{http://scala-sbt.org/}{SBT}
\item
  Biblioteca de I/O:
  \href{http://doc.akka.io/docs/akka/current/scala/io.html}{Akka I/O}
\item
  Buscar ao máximo implementar, sempre que possível, todo o servidor
  utilizando estruturas de dados imutáveis, programação funcional e
  composição de componentes independentes. Este paradigma é fortemente
  suportado pela linguagem Scala, assim como a biblioteca Akka I/O, e
  não necessariamente incorre em perda de eficiência com o devido
  cuidado.
\item
  Focar em simplicidade e elegância arquitetural e correção em vez de
  performance. Devido ao tempo de trabalho e recursos, é incrivelmente
  difícil competir com projetos com vários contribuidores, empresas em
  seu suporte, poder computacional, etc. no quesito performance
\item
  Utilizar um design baseado em ``pipeline'', com os diversos estágios
  de um request sendo tratados independentemente por componentes
  distintos
\end{itemize}

\subsection{Cronograma}\label{cronograma}

O plano do cronograma inicial segue mantido: o trabalho até
outubro/novembro consistirá em implementar o máximo possível do
protocolo em um servidor simples, mas funcional. Para alcançar esse
objetivo, serão necessárias algumas adaptações, como:

\begin{itemize}
\tightlist
\item
  Se limitar a um ou o menor número possível de métodos de negociação\\
\item
  Focar inicialmente na comunicação básica, codificação, decodificação e
  tratamento de headers, assim como funcionamento com streams únicos
  (mas bidirecionais)
\item
  Focar inicialmente na implementação sem TLS (em \emph{plain-text})
\end{itemize}

\hyperdef{}{references}{\label{references}}
\hyperdef{}{ref-raey}{\label{ref-raey}}
Kelsey, John. 2002. ``Compression and Information Leakage of
Plaintext.'' In \emph{Fast Software Encryption}, edited by Joan Daemen
and Vincent Rijmen, 2365:263--76. Lecture Notes in Computer Science.
Springer Berlin Heidelberg.
doi:\url{http://doi.org/10.1007/3-540-45661-9_21}.

\hyperdef{}{ref-pradossl}{\label{ref-pradossl}}
Prado, Angelo, Neal Harris, and Yoel Gluck. 2013. ``SSL, Gone in 30
Seconds.'' In \emph{Black Hat USA}.
\url{http://breachattack.com/resources/BREACH\%20-\%20BH\%202013\%20-\%20PRESENTATION.pdf}.	% No próprio texto é necessário invocar \postextual

\end{document}
